\section{}
\definition{Integral module}{
   We say $B/A$ is \textbf{integral} if every element in $B$ is integral over $A$.

}
\definition{Finite algerbra}{
    $B$ is a finite $A$-algebra if $B$ is finitely generated as an $A$-module.
}
\begin{remark}
    This is stronger than finitely generated algebra. For instance, take $\mathbb{Q}[i]$ over $\mathbb{Q}$. This is finitely generated by $(1,i)$ as a module. 
    But $\mathbb{Q}[x]$ is not finitely generated as a $\mathbb{Q}$ module, but finitely generated as an algebra by $(1,x)$.
\end{remark}

\proposition[]{
    $B$ is a finite $A$-algebra iff $B$ is a finitely generated $A$-algebra and generated by integral elements. 
}
\begin{proof}
    
    ($\implies$). This follows from the previous proposition. $B$ is finitely generated as an $A$-algebra by the same elements. Moreover, each generating element is integral as $B$ is a $A[\alpha]$-module that is finitely generated as $A$-module. This is faithful because $1\in B$ is not killed by $A[\alpha]$ except for $0$.

    ($\impliedby$). Let $B=A[\alpha_1,...,\alpha_n]$ such that each $\alpha_i$ is integral. Then $(\alpha_1,\alpha_1^2,...,\alpha_2,\alpha_2^2,....,\alpha_n,..., \alpha_1\alpha_2,...)$ is finitely generated by finite powers of $\alpha_I$, since each $\alpha_i$ satisfies a monic polynomial, we can pick multi-index $I$ such that it is of degree $\sum \textrm{order of monic polynomial}$. So $B$ is finitely generated as an $A$-module by $(\alpha_1,...,\alpha_1^{k_1},...,\alpha_n,...,\alpha_n^{k_n}, \alpha_1\alpha_2,...)$. 


\end{proof}
\proposition[]{
    $B$ is a finite $A$-algebra iff $B$ is a finitely generated $A$-algebra and generated by integral elements \textbf{and every element in $B$ is integral over }$A$. 
}
\begin{proof}
    Backward direction is the same. For the forward direction, we apply for $b\in B$ the $A[b]$-module $B$ which is finitely generated as an $A$-module and is faithful.
\end{proof}

\proposition[]{
    Let $B$ be an $A$-algebra, and $C$ be a $B$-algebra. Then if $C$ integral over $B$ and $B$ integral over $A$, then $C$ is integral over $A$.
}
\begin{proof}
    Let $c\in C$. Then we have for some $n$ and $b_i\in B$, \[
    c^n + b_{n-1}c^{n-1}+...+b_0 =0 .
    \]
    So $c$ is integral over $B'=A[b_0,...,b_{n-1}]$. Consider the algebra generated by $C'=A[b_0,...,b_{n-1},c]$. This is a finite $B'$ algebra by integral element $c, c^2,...,c^{n-1}$. But $B'$ is also a finite $A$-algebra by previous proposition. So then $C'$ is finite. By the previous proposition, every element in $C'$ is integral over $A$, in particular, $c$ is integral over $A$. 
\end{proof}
\lemma[]{
    If $C$ finite over $B$ and $B$ finite over $A$ as algebras, then $C$ is finite over $A$.
}
\begin{proof}
    Let $\{b_i\}$ generate $B$ as $A$ module and $\{c_j\}$ generate $C$ as $B$ module. Then $\{b_ic_j\}$ generate $C$ as $A$ 
    module. As every $c\in C$ is some \[
    \sum_j r_j c_j = \sum_{i,j} a_{i,j}b_ic_j.
    \]
\end{proof}

\theorem[]{}{
    Let $B$ be an $A$-algebra. Then the set of elements of $B$ that are integral over $A$ is a subalgebra of $B$.

    That is, sum, difference, products of integral elements are integral.
}
\begin{proof}
    Let $\alpha,\beta\in B$ integral over $A$. Consider $A\subseteq A[\alpha,\beta]$. $A[\alpha,\beta]$ is a finite $A$ algebra as it is finitely generated by integral elements. Therefore, we have the stronger statement that every element in $A[\alpha,\beta]$ is integral over $A$.
\end{proof}
\corollary[]{
    The set of elements in $\bar{\mathbb{Q}}$ that satisfy monic polynomials in $\mathbb{Z}[x]$ is a subring.
}
Now let $A$ be in integral domain, and $F$ be its field of fractions.
\definition{Integrally closed/Normal}{
    We say that $A$ is integrally closed/normal if the only elements of $F$ that are integral over $A$ already lie in $A$.
}
\begin{remark}
    For instance, $\mathbb{Z}$ is integrally closed. 
\end{remark}

\theorem[]{}{
    Let $A$ be a unique factorization domain. Then $A$ is integrally closed. In particular, all PID's are integrally closed.
}
\begin{proof}
    Let $\frac{a}{b}\in F$. Let $p$ a prime element such that $p|b$ (so that it does not divide $a$ or else there will be cancellation).
    Suppose that $\frac{a}{b}$ is integral. Then clearing out denominators gives \[
    a^n = b\cdot (\tilde{a})
    \] for some $\tilde{a}\in A$. Because $A$ is a UFD, this is a contradiction as the LHS is not divisible by $p$ but the right is.
\end{proof}

\begin{remark}
    $\mathbb{Z}[2i]$ is not integrally closed. The field of fractions is $\mathbb{Q}[i]$, and $i^2+1=0$ gives an integral element $i\in \mathbb{Q}[i]$ not in $\mathbb{Z}[2i]$.

    Another non example: $\mathbb{C}[t^2,t^3]$. The fraction field is the fraction field of $\mathbb{C}[t]$, for which $t$ is integral as it satisfies $x^2-t^2=0$. 
\end{remark}
\proposition[]{
Let $A$ be normal integral domain, and $F $ be its field of fractions. Let $K/F$ be a finite extension. Then $\alpha\in K$ is integral over $A$ iff the minimal polynomial of $\alpha$ is in $A[x]$.
}
\begin{remark}
    The minimal polynomial is the unique monic polynomial.
\end{remark}
\begin{proof}
    The backwards direction is obvious.
    For the forward direction, suppose that $\alpha\in K$ integral over $A$. Let $f\in A[x]$ kill $\alpha$. Let $g\in F[x]$ be the minimal polynomial. We have $g|f$. Let $K'$ be an extension of $F$ for which $g$ splits. Then set $\alpha_1,...,\alpha_n \in K'$  are the roots. So each $f(\alpha_i)=0$. So each $\alpha_i$ is integral over $A$. So by Vieta's formulae, we have each coefficient in $g$ is a sum of products in the $\alpha_i$. So the coefficients are integral over $A$. Since $A$ is integral, we have $g\in A[x]$.
\end{proof}