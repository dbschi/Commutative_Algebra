\section{}


\proposition[]{
    Let $I\subseteq A$ ideal, $F= \rm{Frac} (A)$. Then if $\alpha \in K/F $ is integral over $I$ the minimal polynomial of $\alpha$ has non-leading coefficients in $r(I)$.
}


\proposition[]{
   Let $B$ integral over $A$. Then $\alpha \in B$ is integral over $I\subseteq A$ iff $\alpha\in r(aB)$.
}




We want to prove the going down theorem.

\theorem[]{Going down}{
    Let $B/A$ integral. \textcolor{red}{Further assume that $A$ is an integral domain and is integrally closed.} Let $q\subseteq B$ prime, $p'\subseteq p \subseteq A$ prime. Also suppose that $q\cap A = p$. Then there exists $q'\subseteq q\subseteq B$ prime such that it contracts to $p'$. 
}


\lemma[]{
    Let $\phi:R_1\to R_2$ ring homomorphism. Let $p_1\subset R_1$ prime. Then there is prime $p_2\subset R_2$ with $\phi^{-1}(p_2)=p_1$ iff $\phi^{-1}(\phi(p_1) R_2)$.
}

\begin{proof}
    We localize at $q$. We want to show $p'B_q\cap A=p'$. Then there is some prime ideal $\tilde{q}\subseteq B_q$ such that $\tilde{q}\cap A= p'$ (by the reverse direction of the previous lemma). $B_q$ is local, so $qB_q$ contains $\tilde{q}$. Take $q'=\tilde{q}\cap B$.

    One inclusion is simple, so we show the $\subseteq $ direction. Let $b\in p'B_q$. Then \[
    b=\frac{y}{s}, y\in p'B, s\in B-q.
    \]
    So $y\in r(p'B)$. So by proposition $b$ is integral over $p'\subset A$. Then the minimal polynomial of $y$ has non leading coefs in $r(p')=p'$. Now let we have $b\in A\cap p'B_q$. Then we must have $b=y/s\implies s=y/b$, as evaluated in $b^{-1}\in \rm{Frac}A$.

    So if $y$ satisfies \[
    y^n+ ...+a_1y+a_0
    \]
    then \[
    s^n+a_{n-1}/b s^{n-1}+...+a_0/b^n=0.
    \]
    Now this is a minimal polynomial of $s$ over $\rm{Frac}A$, as the degree of extension of $y$ and $s$ have to be the same over the fraction field (they are the same extension). Therefore, we can now apply the fact that $s\in B$ is integral over $A$ to get each $a_{n-i}/b^i$ is in $A$. But now we have \[
    \frac{a_{n-i}}{b^i}\cdot b^i = a_{n-i} \in p'.
    \] 
    If $b^i\in p'$ we are done as $p'$ is prime. Else, \[
    s^n = \sum_{i>0}-\frac{a_{n-i}}{b^i} s^{n-i} \in p'B\subseteq q,
    \]
    this contradicts our assumption of $b=\frac{y}{s}\in B_q$.
\end{proof}



\theorem[]{Noetherian Normalization}{
    Let $K$ be a field, $R$ a finitely generated $K$-algebra. Then $R$ is finite over a polynomial algebra.
}
\begin{proof}
    We induct on the number of generators on $R$.

    Let $m$ be the minimal number of generators needed to generate $R$. If $m=1$ we are done as $R=K[\alpha]$. If $\alpha$ is transendental, this is a polynomial ring and we are good. If $\alpha$ is integral, then $R$ is finite over $K$.

    Now we show for $m$-element generated algebras assuming the statement holds for $m-1$ generators.

    Let $R=K[\alpha_1,...,\alpha_m]$. If all of them are algebraically independent, then $R$ is a polynomial ring generated by $\alpha_I$ for each multiindex $I$.
    
    WLOG assume that $\alpha_m$ satisfies a polynomial in coefficients $R_m\defeq K[\alpha_1,...,\alpha_{m-1}]$. If the polynomial is monic $T^d+c_{d-1}T^{d-1}+...+c_1T+c_0$ and each $c_i\in K[\alpha_1,...,\alpha_{m-1}]$, then $\alpha_m$ is integral over this ring. So this $R$ is finite over $R_{m-1}$. Since $R{m-1}$ finite, a finite extension is also finite.

    If the polynomial is not monic, write \[
    f(\alpha_1,...,\alpha_{m-1},T)=\sum_{I,j} c_I \alpha_I T^j, c_I\in K
    \]
    Now we make the change of variables $\alpha_i'=\alpha_i-T^{N^i}$ for big $N>4*({m+\rm{deg}f})$. Then the power of $T$ is uniquely determined by exactly one multiindex (i.e. there are no cancellations).
    Then the leading coefficient of \[
    f(\alpha_i'-T^{N^i},...,T)
    \]
    is some $c_I\in K$, so is essentially a monic polynomial. Now $\alpha_m$ is integral over $K[...\alpha_i-\alpha_m^{N^i},...]$. This is a ring in $m-1$ elements so is finite. The extension is finite so we are done.
\end{proof}