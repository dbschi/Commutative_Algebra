\section{Adjointness}
\renewcommand{\hom}[1]{\rm{Hom}(#1)}
\begin{notation}
    Let $R$ be a ring. We work in $R$-mods.
\end{notation}
\proposition[]{
    Let $A\stackon{$f$}{\to} B \stackon{$g$}{\to} C$.
    Then if $\hom{C,M}\stackon{$\circ g$}{\to} \hom{B,M} \stackon{$\circ f$}{\to} \hom {A,M}$ is exact, the original sequence is exact too.
}
\begin{proof}
    Let $a\in A$. Then pick $M=B/f(a)=B'$, and consider the quotient map $(B\stackon{$\pi$}{\to} B') \in \hom{B,B'}$. Then $\pi \circ f \equiv 0$. So we have a lift $\pi':C\to B'$.
\[\begin{tikzcd}
	& C \\
	B & {B'} & {B/f(A)}
	\arrow["{\pi'}", dashed, from=1-2, to=2-2]
	\arrow["{g}",from=2-1, to=1-2]
	\arrow["\pi"', from=2-1, to=2-2]
	\arrow["{=}"{description}, draw=none, from=2-2, to=2-3]
\end{tikzcd}\]

    But then this means $\pi$ factors through $\rm{ker}(g)$. So that $\rm{Im}(f)\supseteq \rm{ker}(g)$.



    For the other inclusion, take $M=C$. The identity from $C\to C$ maps to $f\circ g$ becomes zero by exactness.  
\end{proof}

\definition[]{Adjoint}{
    Let $\mathcal{C},\mathcal{D}$ categories. Let $F:\mathcal{C}\to \mathcal{D}$ and $G:\mathcal{D}\to \mathcal{C}$ functors. We say $F$ is a left adjoint of $G$ and $G$ is a right adjoint of $F$ if \[
        \hom{F(x),y} \quad \stackon{natural isom.}{\simeq} \quad \hom{x,g(y)}
    \]
}\begin{remark}
    \todo explain natural isomorphism
\end{remark}
\theorem[]{}{
    In the setting of $R$-Mods, $S$-Mods,
    let $F, G$ adjoints. Then $G$ is left exact and $F$ is right exact. 
}\begin{proof}
    $F$ is right exact:
    Let $A\stackon{$f$}{\to} B \stackon{$g$}{\to} C\to 0$. We want $F(A)\to F(B)\to F(C)\to 0$ exact, so want to show that \[
   0 \to \hom{F(C),M}\to \hom{F(B),M}\to \hom{F(A),M}    \]
    exact for every $M\in \mathcal{D}.$

    By adjointness, this is equivalent to saying that 
    \[
    0 \to \hom{C,G(M)}\to \hom{B,G(M)}\to \hom{A,G(M)}  \] is exact for every $M$.


    Let $N=G(M)$. Let $\alpha:C\to N$. Because $g$ is an epimorphism if $\alpha\circ g=0=0\circ g$ then $\alpha=0$. So we have exactness at $\hom{C,G(M)}$.

    For the other exactness, the composite is zero from $g\circ f=0$. Now let $\alpha:B\to N$ such that $\alpha\circ f=0$. Then $\rm{ker}(g)=\rm{Im}(f)\subseteq \rm{ker}(\alpha)$. By the mapping property of quotients we have a lift
\[\begin{tikzcd}
	B & {B/\rm{ker}(g)} & C \\
	& M
	\arrow["g", from=1-1, to=1-2]
	\arrow["\alpha"', from=1-1, to=2-2]
	\arrow["{=}"{description}, draw=none, from=1-2, to=1-3]
	\arrow[dashed,from=1-2, to=2-2]
\end{tikzcd}\]

The other part is left as an exercise.
\end{proof}
\example[]{
    In $R$-mod, $A\to B\to C$ is exact if $\hom{N,A}\to \hom{N,B} \to \hom{N,C}$ is exact for all $N$.



    Moreover, the contravariant functor $\hom{-,N}$ sends right exact sequences to left exact sequences.


    Finally, the covariant functor $\hom{N,-} $ sends left exact sequences to left exact sequences. 
    
    
    For a counterexample, show that exactness is not necessarily preserved in the other cases. 
}
\example[]{
    Show that $\otimes M$ and $\hom{M,-}$ are adjoints. 
}