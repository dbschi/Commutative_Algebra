\section{}
Let $R$ be a local ring with maximal ideal $m$, $M$ a finitely generated $R-$module. 
\corollary[]{
    If the $m_1,...,m_n\in M$ generate $M/m$ as a $R/m$ vector space, then $m_1,...,m_n$ generate $M$ as an $R$ module.
}
\begin{proof}
    This is an application of Nakayama's Lemma. 

    Let $N$ be the module generated by the $m_i$'s. We have \[
    M= N \oplus mM.
    \]
    Now we can mod everything by $N$ to get \[
    M/N = 0 \oplus m M/N.
    \]
    Thus by Nakayama's lemma, $N=M$.
\end{proof}
\theorem[]{}{
    Let $M$ be a non-zero Noetherian $R$ module. 
    Then $\exists$ a filtration by submodules $\{0\}=M_0\subset M_1\subset...\subset M_n=M$ such that $M_{i+1}/M_i=R/p_i$ for some for some prime ideal $p_i$.
}
\definition[]{Annihilator}{
    Let $M$ be an $R$ module. For $m\in M$ we define the annihilator of $m$ \[
    \rm{Ann}(m) \defeq \{r\in R: rm=0\}.
    \]
    This is an ideal.
}
\begin{proof}
    Look at all $\rm{Ann(m)}$ for each $0\neq m\in M$. There is a maximal element in this set by Zorn's lemma. (Exercise) The maximal element is a prime ideal. 

    Let $M_1=R m_1$, where $m_1$ is picked such that annihilator is maximal. This is isomorphic to $R/\rm{Ann}(m_1)$.
    Now we can repeat the process on $M/M_1$ to pick the second prime ideal. This process terminates by Noetherian condition of module. 
\end{proof}


\theorem[]{Krull's Intersection}{
    Let $R$ be a noetherian ring. $I\subset R$ an ideal. Then \[
    I \cap_{n\geq 1} I^n = \cap_{n\geq 1} I^n.
    \]
}
\begin{remark}
    This is not in Atiyah Macdonald, but is in Milne.

    It is tempting to move the $I$ into the intersection, but it is not the same. 
    Counterexample: exercise (smile)
\end{remark}
\begin{proof}
    Let $I=(a_1,...,a_r)$. Then $I^2=(a_ia_j)$, and so on $I^n=(n-\textrm{products of the }a_i)$. The trick now is to notice that this is related to the symmetric polynomials

    \[
    I^n = \{g(a_1,...,a_r): g \textrm{ is a $R$-homogenious polynomial of degree $n$}.\}
    \]
    Let $S_m\defeq \{g(x_1,...,x_r)\in R[x_1,...,x_r]: g(a_1,...,a_r)\in \cap_{a\geq 1} I^n\}$.
    Then \[
    (\bigcup_{m\geq 1}S_m)\subseteq R[x_1,...,x_r]
    \]
    is a finitely generated (generated by ($f_i$)) ideal by Hilbert Basis theorem.

    Let $d$ be such that $f_i\in S_{m_i}$ satisfies $d\geq m_i$.

    Let $b\in \cap I^n$. Then $b\in I^{d+1}$, and we write \[
    b=f(a_1,...,a_r),
    \]
    $f\in S_{d+1}$, so can be written as \[
    f=\sum_i g_i f_i.
    \] 
    We can pick $g_i$ such that these are homogeneous with degree $\rm{deg}\ f-\rm{deg}\  f_i> 0$. If not, the different degrees have to cancel each other. So the $g_i$ have no constant terms. 
    Evaluate these at the $a_i$'s. Since $g$ has no constant term, $g(a_i)\in I$ and we have expressed $f$ in the left hand side.
\end{proof}
\subsection*{Localization}
\begin{adefinition}{Multiplicatively Closed Set}{}
    $S\subseteq R\backslash\{0\}$ is multiplicatively closed if it contains $1$ and $s,t\in S \implies st\in S$.

    We define localization \[
    S^{-1}R\defeq \{\left[\frac{r}{s}\right]: r\in R,s\in S\}/ \sim,
    \]
    were the equivalence relation is \[
    \frac{r_1}{s_1}\sim \frac{r_2}{s_2} \textrm{ if } (r_1s_2-r_2s_1)\cdot s=0
    \]
    for some $s\in S$.
\end{adefinition}
\proposition[]{
    $S^{-1}R $ is a ring by the standard definitions plus and minuses.\begin{align*}
        \frac{r_1}{s_1} + \frac{r_2}{s_2}=\frac{r_1s_2+r_2s_1}{s_1s_2}\\
        \frac{r_1}{s_1}\cdot\frac{r_2}{s_2}=\frac{r_1r_2}{s_1s_2}.
    \end{align*}
    It has additive and multiplicative identity $\frac{0}{1}, \frac{1}{1}$ respectively.
}
\proposition{
    There is a canonical ring homomorphism $R\to S^{-1}R$. This sends $r\to \frac{r}{1}$.
}
\theorem[]{Universal property of localization}{
    Let $g:R\to R'$ be a ring homomorphism such that $g(s)$ is a unit for every $s\in S$. Then there is a unique map $g_s S^{-1}R\to R'$ such that the composite $R\to S^{-1}R\to R'$ is equal to $g$.

    In other words, $S^{-1}R$ is the smallest ring such that every element in $S$ is a unit.
}
