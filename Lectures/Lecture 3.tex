\section{}

\theorem[]{}{
    Let $R$ be a ring. Let $P_1,..., P_n\subseteq R$ be prime ideals. Let $I \subseteq \cup P_i$ be an ideal. Then $I$ is contained in some $P_i$.
}
\begin{remark}
    There is a counterexample. In $\mathbb{F}_2[x,y]$ pick $I=(x,y)$. We can find three ideals in $\mathbb{F}_2[x,y]$ whose union contains $(x,y)$ but none contains $(x,y)$. (left as an ``interesting'' exercise)
    The same is not true for an infinite field.
\end{remark}
\begin{proof}
    We induct on $n$. $n=1$ is easy. We look at the case for $n=2$ as an example.
    Let $n=2$.
    We suppose that $I$ is not contained in either $P_1$ or $P_2$.
    Suppose $a_1,a_2\in I$ such that $a_1\notin P_1$, $a_2\notin P_2$ (so that $a_1\in P_2$, $a_2\in P_1$). Then $a_1+a_2\in I$ is not an element of $P_1$ or $P_2$. This gives a contradiction.

    The idea is to pick element in $I\cap P_{k\neq i}$ for each $i$ from $1-n$.

    Suppose the statement holds for $n-1$, we want to show for $n$. Then by contradiction suppose $I$ is not contained in either of the $P_i$'s. Then by the induction hypothesis we can assume that there are no inclusion among the $P_i$'s, since this will reduce to the case for $n-1$.
    Choose elements $a_i \in I$ but $a_i\notin P_i$. Then for each other $P_{j\neq i}$ pick an element $b_k$ distinct from $P_i$ and multiply $a_i$ by $b_k$. Then this product of $b_k$'s and $a_i$ is not in $P_i$, as $P_i$ is prime. However, it is in the intersection of $I$ and the $P_{k\neq i}$'s. The sum of all the products $b_{k\neq i} a_i$ is not in each of the ideals.  
\end{proof}
\definition{Coprime ideals}{
    Let $I_1,I_2\subseteq R$. Then they are coprime $I_1+I_2=R$. 
}
\proposition{
    Let $I_1, I_2\subseteq R$. Let  \[
        I_1\cdot I_2 \defeq (ab: a\in I_1, b\in I_2).
        \] 
        We have \[
        I_1\cdot I_2\subseteq I_1\cup I_2,
        \]
        with equality when the ideals are coprime.
}
\begin{remark}
    The equality condition of coprime is not an if and only if in the first statement. For example the $0$ ideal plus the $0$ ideal does not have $1$. 
    The algebraic completion $\bar{\mathbb{Z}}\subseteq \bar{\mathbb{Q}}$ is a non-noetherian subring and contains $p,p^{1/2},p^{1/3}...$. The ideal generated by $I={p^{a/b}: a/b \textrm{ is a positive real number}}$ satsifies $I^2=I=I\cap I$.
\end{remark}
\begin{proof}
    We prove the specific statement first. The first inclusion is obvious as $ab\in I_1\cap I_2$. For the other inclusion, let $I,J$ coprime ideals
    Pick $i\in I, j\in J$ such that $i+j=1$.
    Then for every $a\in I\cap J$ we have \[
    a=a\cdot 1 = ai + aj
    \]
    is a sum of an element in $I$ and an element in $J$.
\end{proof}
\theorem[]{Chinese Remainder}{
    Let $I,J$ be coprime ideals. Then \[
    R/(I\cap J) \to R/I \oplus R/J
    \]
    is an isomorphism.

    In general, let $I_1,...,I_n$ be ideals of $R$ such that they are pairwise coprime.
    Then we have \[
    R/\cap_i I_i \to \oplus R/I_i
    \]is an isomorphism.
}
\begin{proof}
    We prove the case for two ideals. The case for multiple ideals is an exercise. 
    We have a natural ring morphism from $R\to R/I \oplus R/J$. So we want to show that this is surjective with kernel $I\cap J$. 

    The kernel is $I\cap J$ by definition as $x=0\mod I$ and $x=0\mod J$ iff $x\in I$ and $x\in J$.

    For surjection, pick $i\in I$, $j\in J$ such that $i+j=1$. 
    Then $i=1 \mod J$ and $j=1 \mod I$
    Then for every $([a],[b])\in R/I\oplus R/J$,
    $aj+bi$ maps to this element by linearity. 
\end{proof}

\theorem[]{Nakayama's Lemma}{
    Let $J\subseteq J(R)$ be an ideal. Let $M$ be a finitely generated $R$ module such that $JM=M$.
    Then $M=0$.
}
\begin{proof}
    Let $(e_1,...,e_n)$ be a minimal set of generators for $M$.
    Then we have \[
    m_1=j_1m_1+j_2m_2+...+j_nm_n.
    \]
    Such that $j_i\in J\subseteq J(R)$.
    But then\[
    (1-j_1)m_1= j_2m_2+...+j_nm_n.
    \]
    Because $1-j_1$ is a unit by the characterization of Jacobson radical, we are done.
\end{proof}
\begin{remark}
    The thing fails if $M$ is not finitely generated. Let $R$ be the set of fractions $\{a/b: p \textrm{ does not divide }b\}$. Then $(p)$ is the unique maximal ideal. If we take $M=\mathbb{Q}$, we have $(p)M =M$. 
\end{remark}