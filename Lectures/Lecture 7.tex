\section{}

\begin{remark}
    Atiyah Macdonald is so terse... 
\end{remark}

We continue the discussion of Open sets of Zarski topology.

\proposition[]{
    Let $\spec R=\cup_{i\in I} X_i$, where each $X_i$ open. Then there is a finite cover\[
    \spec R = X_1\cup ...\cup X_n.
    \]
}
\begin{proof}
    We assume that $X_i=D(f_i)$ since $D$ forms a basis.
    We claim that $(f_i)_{i\in I}=R$. 
    Suppose not, then there is some prime (maximal) ideal $p$ containing every $f_i$. But this is absurd because then we would not have a covering. Write $1=\sum_{i} a_if_i$. Then We would have a finite covering.
\end{proof}
\definition[]{}{
    Let $\phi:A\to B$ be a ring homomorphism. Since each prime ideal's preimage is prime, we can define a map induced by $\phi$\[
    \phi^*:\spec B\to \spec A.
    \]
}
\proposition[]{
    $\phi^*$ is continuous.
}
\begin{proof}
    Let $V(I)$ be closed in $\spec A$. We want $\phi^{*{-1}} (V(I))$ be closed. We just check that $\phi^{*{-1}}(V(I))=V(\phi(I))$ 
\end{proof}

\example[]{
    Let $f\in R$. We have a ring homomorphism $\phi_f:R\to R_f$. Show that the map \[
    \phi^*_f:\spec R_f \to \spec R
    \] is a homeomorphism onto $D_f$.
}
\begin{proof}
    In exercise 2.
\end{proof}
\newcommand{\ffrac}{\rm{Frac}}
Let $x\in \spec R$. Consider the $R_x\defeq \ffrac(R/p_x)$, the field of fractions of the integral domain. Given $f\in R$ we can define a map that sends \[
x\mapsto f\mod p_x.
\]
\example[]{
    In $\mathbb{C}[x,y]$, take a maximal ideal $(x-a,y-b)$. Take $f$ a polynomial, then the map\[
    (x-a,y-b) \mapsto f(a,b)
    \]
    is the evaluation map.

    Now take $I\subseteq \mathbb{C}[x,y]$ a radical ideal i.e. if $a^k\in I$ then $a\in I$.
    We now evaluate this on $V(I)$. 
    a function $f$ induces the identically zero map if and only if it is nilpotent. Similarly, the function $f$ is identically zero on $V(I)$ if and only if $f\in r(I)=I$. So $f\in I$. So if two functions are in the same coset $\mathbb{C}[x,y]/I$, they induce the same function on $V(I)$. So we can view functions on $V(I)$ as functions onto $R/I$.
}

\example[]{
    Let $I=r(I)$. 
Let $\phi : R\to R/I$. Show that $\phi^*:\spec(R/I) \to \spec R$ is a homeomorphism onto $V(I)$.
}


\definition{Integral}{
    Let $B$ be an $A$-algebra. We say that $\alpha \in B$ is \textbf{integral} over $A$ if $\alpha$ satisfies\[
        \alpha^n + a_{n-1}\alpha^{n-1}+...+a_1\alpha + a_0=0.
    \]
    for some $a_i\in A$.
}
\example[]{
    $1/2\in \mathbb{Q}$ is not integral in $\mathbb{Z}$. This is because the polynomial has to be a monic.

    $i\in \mathbb{C}$ is integral over $\mathbb{Z}$. 
}
We want to prove the following result:
\theorem[]{}{
    If two elements are integral then their sum and products are integral. 
}
\proposition[]{
    The following are equivalent. 

    \begin{enumerate}
        \item $\alpha\in B$ is integral over $A$.
        \item There is a faithful $A[\alpha]\subseteq B$ submodule that is finitely generated as an $A$ module.
    \end{enumerate}
}
\begin{remark}
    An $R$-module $M$ is faithful if $r_1m=r_2m$ for all $m\in M\implies r_1=r_2$.  
\end{remark}

\begin{proof}
    $1\implies 2:$ Let $\alpha\in B$ be integral. Such that $\alpha^n + a_{n-1}\alpha^{n-1}... +a_0$. Consider the submodule $M=A[\alpha]$. It contains $1_M$ and we have $r\in A[\alpha]$ satisfies $r\cdot 1=r_m$. So this module is faithful.

    Now notice that $A[\alpha]$ is generated by $\{1,\alpha,...,\alpha^{n-1}\}$ by the monic polynomial relation.

    $2\implies 1$. Let $e_1,...,e_n$ be a generating set (over $A$). Then \[
        \alpha e_i \in \rm{Span}_A \{e_i\}.
    \]
    Then there is some matrix transformation \[
    (M-\alpha I_n)\begin{bmatrix}
        e_1 \\ e_2 \\ \vdots \\ e_n
    \end{bmatrix}=\vec{0}.
    \]
    Multiply by the adjucate of $(M-\alpha I_n)$ gives \[
    \det (M-\alpha I_n) I_n \begin{bmatrix}
        e_1 \\ e_2 \\ \vdots \\ e_n
    \end{bmatrix} =\vec{0}
    \]
    kills every element in 
    By faithfulness this has to be the zero transformation, so that \[
    \det (M-\alpha I_n) = 0 \in A[\alpha].
    \]
    The characteristic polynomial is a monic (up to factor of $-1$) polynomial in $\alpha$, so we are done.

\end{proof}