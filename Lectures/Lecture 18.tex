\section{Exam review}

\begin{aexample}{Computing Krull Dimension}{}
    The Krull dimension of $K[x_1,...,x_n]$ is $n$.
    This is because we have a maximal chain \[
    (0)\subset (x_1) \subset(x_1,x_2)\subset...\subset (x_1,...,x_n)\subset K[x_1,...,x_n].
    \]
    This is because $(x_1)$ is principal, and if there is a minimal prime ideal within this, then $x_1$ would be a multiple of the generator.

    Now we mod everything by $(x_1)$, in $K[x_1,...,x_n]/(x_1) \simeq K[x_2,...,x_n]$ we get the ideals \[
    (0)\subset (x_2)\subset (x_2,x_3)...
    \]
    inductively $x_2$ is minimal prime ideal. 


    Another algebra: $K[x_1,...,x_n]/(x_1^2+...+x_n^2)$ for $n\geq 3$. The images of $x_1,...,x_{n-1}$ are algebraically independent. However, $x_1,...,x_n$ is not algebraically independent. 

    If you mod out by more quotients, we would expect with high probability that each polynomial reduces the transendence degree by $1$.
\end{aexample}

\begin{remark}
    The next homework should not include exam material, probably. 
\end{remark}
\begin{remark}
    You should be need to prove the intersection of theorems of the prelim syllabus and the theorems we have seen in class.
\end{remark}

\example{
    Look at $\max\spec K[x_1,...,x_n]\simeq K^n$. Now if $R$ integral over $K[x_1,..,x_n]$ we have a surjection into $K^n$ with finite fibers. So $\max\spec R$ is somewhat $n$-dimensional. 
}

\example[]{
    Homeomorphism between spec is not enough to show homeomorphism between rings. Take two fields $F_1,F_2$ of the same cardinality (but not characteristics). The spec $F_1[x]$ is $p_1+F_1$, resp. $p_2+F_2$. The open sets are generated by $p+ \textrm{ finitely many points }$. Now take any bijection between $F_1$ and $F_2$. 
}