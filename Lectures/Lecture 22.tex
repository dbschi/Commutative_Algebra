\section{}
\proposition[]{
    $\otimes N$ and $\hom{N,-}$ are adjoints. 
}
\begin{proof}
    Consider this pair of morphisms. We can check that they are inverses of each other and are well defined, and are natural.
    \[\begin{tikzcd}
	{\rm{Hom}(A\otimes N,B)} && {\rm{Hom}(A,\rm{Hom}(N,B))} \\
	f && {[a\mapsto (n\mapsto f(a\otimes n))]} \\
	{a\otimes n\mapsto g(a)(n)} && g
	\arrow["\simeq"', from=1-1, to=1-3]
	\arrow[maps to, from=2-1, to=2-3]
	\arrow[maps to, from=3-3, to=3-1]
\end{tikzcd}\]
\end{proof}
\corollary[]{
    $\otimes M$ is right exact. $\hom{M,-}$ is left exact.
}


\example[]{
    Tensoring is not left exact.

    \[
    0\to \mathbb{Z}\stackon{$\times 2$}{\to}\mathbb{Z}
    \]
    tensored with $\mathbb{Z}/2$ gives \[
     0\to \mathbb{Z}/2\stackon{$\times 0$}{\to}\mathbb{Z}
    \]
    is not exact.
}

\definition{Free resolution}{
    Let $M$ be an $R$ module. A \textbf{free resolution} is an exact sequence \[
    ...\to F_2\to F_1 \to M\to 0
    \]
    such that each $F_i$ is free.
}
\begin{remark}
    In fact we only need $F_i$'s to be projective. 
\end{remark}
\begin{adefinition}{Derived Functor (Tor)}{}
    
    Let $M,N$ be $R$-modules. Take a free resolution of $M$ and tensor with $N$.

    \[\begin{tikzcd}
	{...} & {F_2} & {F_1} & M & 0 \\
	{...} & {F_2\otimes N} & {F_1\otimes N} & {M\otimes N} & 0
	\arrow[from=1-1, to=1-2]
	\arrow[""{name=0, anchor=center, inner sep=0}, from=1-2, to=1-3]
	\arrow[from=1-3, to=1-4]
	\arrow[from=1-4, to=1-5]
	\arrow[from=2-1, to=2-2]
	\arrow[""{name=1, anchor=center, inner sep=0}, from=2-2, to=2-3]
	\arrow[from=2-3, to=2-4]
	\arrow[from=2-4, to=2-5]
	\arrow["{\otimes N}", shorten <=4pt, shorten >=4pt, Rightarrow, from=0, to=1]
\end{tikzcd}\]

The $i$-th homology of this complex is the $\rm{Tor} ^i$, with the exception that $\rm{Tor}^0$ is the homology at $F_2\otimes N\to F_1\otimes N \to 0$ which is $M\otimes N$ by the right exactness of tensoring.

\end{adefinition}
\begin{remark}
    Tor is symmetric and distributes over direct sums. This is because the original functor ($\otimes N$) also has these properties.
\end{remark}
\begin{aexample}{Tor for PIDs}{}
    Let $R$ be a PID. $M$ a finitely generated $R$-module. Then write \[
    M = M^{\rm{free}}\oplus M^{\rm{torsion}}
    \]
    where\[
    M^{\rm{torsion}}\simeq \oplus R/(p_i^{a_i}).
    \]

    We have $\rm{Tor}^i(M^{\rm{free}},N)=0$ for $i\geq 1$. This is because it is already a free resolution.
    
    On the torsion part, we work with each $R/(p^a)$ piece separately. 

    We get the free resolution \[
    ...\to 0\to R \stackon{$\times p^a$}{\to} R \to R/a \to 0
    \]
    We tensor this with $N$. $\rm{Tor}^2 (R/(p^a),N)$ and above vanish. The special case is $\rm{Tor}^1$ which is the kernal of $R\otimes N \stackon{$\times p^a$}{\to} R\otimes N$.

    But as $N$ is already an $R$ module, we have isomorphisms\[\begin{tikzcd}
	{r\otimes n} & {R\otimes N} & {R\otimes N} \\
	{r\cdot n} & N & N
	\arrow[maps to, from=1-1, to=2-1]
	\arrow["p^a", from=1-2, to=1-3]
	\arrow["\simeq"', from=1-2, to=2-2]
	\arrow["\simeq", from=1-3, to=2-3]
	\arrow["p^a"', from=2-2, to=2-3]
\end{tikzcd}\]
So that $\rm{Tor}^1(R/(p^a),N) = \{n\in N : p^an=0\}$.


The same argument goes through for $\rm{Tor}(R/(a),N)$ for any principal ideal $a$.
\end{aexample}
\begin{aproposition}{}{}
    Let $R$ be an integral domain, $a\in R$. Then as $R$-modules \[
    \rm{Tor}^1(R/(a), N)=\{n\in N:an=0\},
    \]
    and $0$ for degrees $2$ and above.
\end{aproposition}

\example[]{
    Over $\mathbb{Z}/8$ modules, a free resolution of $\mathbb{Z}/2$ is \[
    ... \stackon{$\times 2$}{\to} \mathbb{Z}/8 \stackon{$\times 4$}{\to}\mathbb{Z}/8 \stackon{$\times 2$}{\to}\mathbb{Z}/8\to \mathbb{Z}/2 \to 0
    \]
    Tensoring with $N$ would be multiplication by $2$ and $4$ (alternating).
}