\section{}

\example[]{
    Let $I\subseteq R$ ideal.
    For every $n\in \mathbb{N}$ let $R_n\defeq R/I^n$.
    
    We have a directed set indexed by $\mathbb{N}$ and (natural) projections \[
    R/I^m\stackon{$\pi_{m,n}$}{\to} R/I^n
    \]
    for $m\geq n$.


    Then the inverse limit is \[
    \ilim_n R_n \defeq \hat{R}^{/I} \subseteq \prod_n R_n
    \]
    such that each element $(...,x_2,x_
1,x_0)$ satisfies $\pi_{m,n}(x_m)=x_n$ for $m\geq n$.
}
We restate the universal property of the inverse limit (Proposition \ref{prop:ilimuniversal}).
\begin{aproposition*}{Universal property of Inverse Limit}{}
    The inverse limit is a limit in the slice category. I.e. it satisfies the universal property of being a terminal object.

    Concretely, let $A$ be any ring with homomorphisms $f_n:A\to R_n$ commuting with $\pi_{m,n}$. Then there is a unique map of $f:A\to \hat{R}^{/I}$ such that $f_n=\pi_n\circ f$.
\end{aproposition*}
\begin{proof}
    The inverse limit is evidently a ring with addition and multiplication entry-wise.

    The universival property is left as exercise... \textcolor{red}{TODO}.
\end{proof}
\corollary[]{
    There is a canonical map from $R\to \hat{R}^{/I}$.

    This map is given by \[
    r\mapsto (...,r \mod I^n, ...)
    \]
    and has kernel \[
    \bigcap_n I^n.
    \]
}

\begin{remark}
    The map is not necessarily surjective. 


    Consider $\mathbb{Z}\subset \mathbb{Z}_{(p)}$. Then we have the diagram% https://q.uiver.app/#q=WzAsNCxbMCwwLCJcXG1hdGhiYntafSJdLFswLDEsIlxcbWF0aGJie1p9X3socCl9Il0sWzEsMCwiXFxtYXRoYmJ7Wn0vcF5uIl0sWzEsMSwiXFxtYXRoYmJ7Wn1feyhwKX0vcF5uIl0sWzAsMSwiIiwyLHsic3R5bGUiOnsidGFpbCI6eyJuYW1lIjoiaG9vayIsInNpZGUiOiJ0b3AifX19XSxbMCwyXSxbMSwzXSxbMiwzLCJcXHNpbWVxIl1d&macro_url=%5Ccdo
\[\begin{tikzcd}
	{\mathbb{Z}} & {\mathbb{Z}/p^n} \\
	{\mathbb{Z}_{(p)}} & {\mathbb{Z}_{(p)}/p^n}
	\arrow[from=1-1, to=1-2]
	\arrow[hook, from=1-1, to=2-1]
	\arrow["\simeq", from=1-2, to=2-2]
	\arrow[from=2-1, to=2-2]
\end{tikzcd}\]

So the natural inclusion into the localization is a morphism in the slice category. Thus we have % https://q.uiver.app/#q=WzAsMyxbMCwwLCJcXG1hdGhiYntafSJdLFswLDEsIlxcbWF0aGJie1p9X3socCl9Il0sWzEsMCwiXFxtYXRoYmJ7Wn1fcD1cXHZhcnByb2psaW1fbiBcXG1hdGhiYntafS9wXm4iXSxbMCwxXSxbMCwyXSxbMSwyXV0=&macro_url=%5Ccdo
% https://q.uiver.app/#q=WzAsNCxbMCwwLCJcXG1hdGhiYntafSJdLFswLDEsIlxcbWF0aGJie1p9X3socCl9Il0sWzEsMCwiXFxtYXRoYmJ7Wn1fcCJdLFsyLDAsIlxcdmFycHJvamxpbV9uIFxcbWF0aGJie1p9L3BebiJdLFswLDFdLFswLDJdLFsxLDJdLFsyLDMsIj0iLDEseyJzdHlsZSI6eyJib2R5Ijp7Im5hbWUiOiJub25lIn0sImhlYWQiOnsibmFtZSI6Im5vbmUifX19XV0=&macro_url=%5Ccdo
\[\begin{tikzcd}
	{\mathbb{Z}} & {\mathbb{Z}_p} & {\varprojlim_n \mathbb{Z}/p^n} \\
	{\mathbb{Z}_{(p)}}
	\arrow[from=1-1, to=1-2]
	\arrow[from=1-1, to=2-1]
	\arrow["{=}"{description}, draw=none, from=1-2, to=1-3]
	\arrow[from=2-1, to=1-2]
\end{tikzcd}\]


In general, this is even bigger than the localization of $\mathbb{Z}$.
Hensel's lemma gives a compatible sequence of numbers $b_n\in \mathbb{Z}/19^n\mathbb{Z}$ with $b_n^2\equiv 6\mod 19^n$. Thus $\mathbb{Z}_p$ has irrational numbers.
\end{remark}



\definition{$I$-adic topology}{
    Fix $I\subseteq R$ ideal.
We give $\hat{R}^{/I}$ a topological ring structure. We give a basis of neighborhoods around $0$, then the translation $+r$ gives a basis of neighborhoods around $r$ for each $r$.
The union of all these generates a topology known as the \textbf{$I$-adic topology}. 


Concretely, the basis around $0$ is \[
\{\{\ker \hat{R}^{/I}\to R/I^n\}\}.
\]
} 

\proposition[]{
    $\hat{R}^{/I}$ is complete with respect to this topology.
}\example[]{
    As an example of `completeness', consider $R=\mathbb{Z},I=(p)$. The topology on $\mathbb{Z}_p$ is induced by a metric (two numbers are close if their difference is a multiple of a big power of $p$). Each Cauchy sequence may not converge in $\mathbb{Z}$ but will converge in $\mathbb{Z}$.
}

\todo spec of $\mathbb{Z}_p$.
