\section{Hom and Ext}
%god its so fucking cold i need a coat
    Let $M$ be an $R$-module. The functor $\rm{Hom}(M,-)$ is covariant and the functor $\rm{Hom}(-,M)$ is contravariant.

Both are left exact (see Propositions \ref{prop:contraexact} and \ref{prop:coexact}).


\begin{adefinition}{Injective modules}{}
    $I$ is injective if there exists a dotted arrow for every solid diagram of the following.
    % https://q.uiver.app/#q=WzAsNCxbMSwxLCJBIl0sWzIsMSwiQiJdLFswLDEsIjAiXSxbMSwwLCJJIl0sWzAsMSwiIiwwLHsic3R5bGUiOnsidGFpbCI6eyJuYW1lIjoiaG9vayIsInNpZGUiOiJ0b3AifX19XSxbMiwwXSxbMCwzXSxbMSwzLCIiLDEseyJzdHlsZSI6eyJib2R5Ijp7Im5hbWUiOiJkYXNoZWQifX19XV0=
\[\begin{tikzcd}
	& I \\
	0 & A & B
	\arrow[from=2-1, to=2-2]
	\arrow[from=2-2, to=1-2]
	\arrow[hook, from=2-2, to=2-3]
	\arrow[dashed, from=2-3, to=1-2]
\end{tikzcd}\]

\end{adefinition}
\begin{remark}
    Free modules need not be injective. 
    There is an embedding of $\mathbb{Z}$ into $\mathbb{Q}$ but the identity map $\mathbb{Z}\to \mathbb{Z}$ does not lift to $\mathbb{Q}\to \mathbb{Z}$.

    In general for an integral domain $R$, % https://q.uiver.app/#q=WzAsNCxbMCwxLCIwIl0sWzEsMSwiUiJdLFsyLDEsIlxccm17RnJhY30oUikiXSxbMSwwLCJJIl0sWzAsMV0sWzEsMiwiMVxcbWFwc3RvIHIiLDJdLFsxLDMsIjFcXG1hcHN0byBuIl0sWzIsMywiMVxcbWFwc3RvIG4nIiwyLHsic3R5bGUiOnsiYm9keSI6eyJuYW1lIjoiZGFzaGVkIn19fV1d
\[\begin{tikzcd}
	& I \\
	0 & R & {\rm{Frac}(R)}
	\arrow[from=2-1, to=2-2]
	\arrow["{1\mapsto n}", from=2-2, to=1-2]
	\arrow["{1\mapsto r}"', from=2-2, to=2-3]
	\arrow["{1\mapsto n'}"', dashed, from=2-3, to=1-2]
\end{tikzcd}\]
means that we need $n'\in I$ such that $rn'=n$.
\end{remark}
\begin{adefinition}{Divisible}{}
    Let $R$ be an integral domain. $I$ is divisible if for every $n\in I$, $r\in R\backslash\{0\}$, there is $n'\in I$ s.t. $rn'=n$.
\end{adefinition}\proposition[]{
    Let $R$ be integral domain. Then injective modules are divisible.
}
\example[]{
    As $\mathbb{Z}$ modules, $\mathbb{Z}^k, \mathbb{Z}/n$ modules are not divisble. 


    $\mathbb{Q}^k, \mathbb{Q}/\mathbb{Z}$ are divisible. 
}

An injective resolution of $\mathbb{Z}/n$ is
\[\begin{tikzcd}
	0 & {\mathbb{Z}/n} & {\mathbb{Q}/\mathbb{Z}} & {\mathbb{Q}/\mathbb{Z}} & 0 \\
	& {\frac{1}{n}\mathbb{Z}/\mathbb{Z}}
	\arrow[from=1-1, to=1-2]
	\arrow[from=1-2, to=1-3]
	\arrow["\simeq"{description}, draw=none, from=1-2, to=2-2]
	\arrow["n"', from=1-3, to=1-4]
	\arrow[from=1-4, to=1-5]
\end{tikzcd}\]
\begin{adefinition}{Projective Module}{}
    $P$ is projective if there is a dotted arrow for every solid diagram 
    % https://q.uiver.app/#q=WzAsNCxbMSwxLCJBIl0sWzIsMSwiQiJdLFswLDEsIjAiXSxbMSwwLCJQIl0sWzEsMCwiIiwyLHsic3R5bGUiOnsiaGVhZCI6eyJuYW1lIjoiZXBpIn19fV0sWzAsMl0sWzMsMF0sWzMsMSwiIiwxLHsic3R5bGUiOnsiYm9keSI6eyJuYW1lIjoiZGFzaGVkIn19fV1d
\[\begin{tikzcd}
	& P \\
	0 & A & B
	\arrow[from=1-2, to=2-2]
	\arrow[dashed, from=1-2, to=2-3]
	\arrow[from=2-2, to=2-1]
	\arrow[two heads, from=2-3, to=2-2]
\end{tikzcd}\]
\end{adefinition}

\proposition[]{
    $I$ is projective $\iff $ Hom $(I,-)$ is exact.

    Similarly $P$ is injective $\iff$ Hom$(-,P)$ is exact
}
\begin{proof}
    Both functors fail to be exact at the right side. But the last map of Hom's being surjective is equivalent to saying that the module is projective/injective.
\end{proof}

Ext measures the failure of Hom$(M,-)$ and Hom$(-,N)$ to be exact.
\begin{adefinition}{Ext functor}{}
    Let $M,N$ modules, use a free resolution of $M$ and apply with Hom$(-,N)$ functor.
    % https://q.uiver.app/#q=WzAsNixbMSwwLCJQXzIiXSxbMiwwLCJQXzEiXSxbMywwLCJQXzAiXSxbMCwwLCIuLi4iXSxbNCwwLCJNIl0sWzUsMCwiMCJdLFswLDFdLFsxLDJdLFszLDBdLFsyLDRdLFs0LDVdXQ==
% https://q.uiver.app/#q=WzAsMTIsWzEsMCwiUF8yIl0sWzIsMCwiUF8xIl0sWzMsMCwiUF8wIl0sWzAsMCwiLi4uIl0sWzQsMCwiTSJdLFs1LDAsIjAiXSxbMywxLCJcXHJte0hvbX0oUF8wLE4pIl0sWzIsMSwiXFxybXtIb219KFBfMSxOKSJdLFsxLDEsIlxccm17SG9tfShQXzIsTikiXSxbNCwxLCJcXHJte0hvbX0oTSxOKSJdLFs1LDEsIjAiXSxbMCwxLCIuLi4iXSxbMCwxXSxbMSwyXSxbMywwXSxbMiw0XSxbNCw1XSxbNiw3XSxbNyw4XSxbOSw2XSxbMTAsOV0sWzgsMTFdLFsxMywxNywiXFxybXtIb219KC0sTikiLDAseyJzaG9ydGVuIjp7InNvdXJjZSI6MjAsInRhcmdldCI6MjB9fV1d
\[\begin{tikzcd}
	{...} & {P_2} & {P_1} & {P_0} & M & 0 \\
	{...} & {\rm{Hom}(P_2,N)} & {\rm{Hom}(P_1,N)} & {\rm{Hom}(P_0,N)} & {\rm{Hom}(M,N)} & 0
	\arrow[from=1-1, to=1-2]
	\arrow[from=1-2, to=1-3]
	\arrow[""{name=0, anchor=center, inner sep=0}, from=1-3, to=1-4]
	\arrow[from=1-4, to=1-5]
	\arrow[from=1-5, to=1-6]
	\arrow[from=2-2, to=2-1]
	\arrow[from=2-3, to=2-2]
	\arrow[""{name=1, anchor=center, inner sep=0}, from=2-4, to=2-3]
	\arrow[from=2-5, to=2-4]
	\arrow[from=2-6, to=2-5]
	\arrow["{\rm{Hom}(-,N)}", shorten <=4pt, shorten >=4pt, Rightarrow, from=0, to=1]
\end{tikzcd}\]
We define \[
\rm{Ext}^i(M,N)\defeq H_i
\]
the i-th degree (co)homology of this complex. 
\end{adefinition}
\begin{remark}
    $\rm{Ext}^0(M,N)=\rm{Hom}(M,N)$ by exactness of the functor.

    The result is also the same by taking an injective resolution of $N$  \[
    0\to N \to I_0\to I_1 \to...
    \]
     and applying $\rm{Hom}(M,-)$.
\end{remark}

\begin{atheorem}{Baer's Criterion}{}
    Let $I$ be an $R$-module.
    The following are equivalent \begin{enumerate}
        \item $I$ is an injective module.
        \item For any ideal $a\subseteq R$, we have a lift % https://q.uiver.app/#q=WzAsNCxbMCwxLCIwIl0sWzEsMSwiYSJdLFsyLDEsIlIiXSxbMSwwLCJJIl0sWzAsMV0sWzEsMiwiXFxzdWJzZXRlcSIsMSx7InN0eWxlIjp7ImJvZHkiOnsibmFtZSI6Im5vbmUifSwiaGVhZCI6eyJuYW1lIjoibm9uZSJ9fX1dLFsyLDMsIiIsMix7InN0eWxlIjp7ImJvZHkiOnsibmFtZSI6ImRhc2hlZCJ9fX1dLFsxLDNdXQ==
\[\begin{tikzcd}
	& I \\
	0 & a & R
	\arrow[from=2-1, to=2-2]
	\arrow[from=2-2, to=1-2]
	\arrow["\subseteq"{description}, draw=none, from=2-2, to=2-3]
	\arrow[dashed, from=2-3, to=1-2]
\end{tikzcd}\]
    \end{enumerate}
\end{atheorem}

\begin{proof}
    $1\implies 2$.

    We now show $2\implies 1$.
    Suppose $I$ satisfies $2$. 
    Let $A\stackon{$f$}{\to} I$, and $A$ injects into $B$.
    Let $S=\{A',f': A\subseteq A'\subseteq B, f':A'\to I \textrm{ extends } f\}$, with partial order by inclusion. Every chain has an upper bound so we can apply zorn's lemma to find a maximal $A'\subseteq B $, $f':A'\to I$ that extends $A$. We want to show $A'=B$. Suppose not, then let $b\in B$ that is not in $A'$. 

    Let $a=\{r\in R: rb\in A'\}$. 

    We have a map $a\stackon{$g'$}{\to} I$ that sends $r\mapsto f'(rb)$. This is well defined because $rb\in A'$ and can be lifted (by hypothesis) to $R\to I$.

    Define \[
    h:A' + Rb \to I
    \]
    by $h(a_0 + rb) = f'(a_0)+g'(r)$.

    We can check that this is well defined and extends $f'$, which is a contradiction. 
\end{proof}

\corollary[]{
    Let $R$ be a principal ideal domain and $I$ be divisible. Then $I$ is injective.
}
\begin{proof}
    All ideals are of the form $(r)$. This lifts by divisibility. Now apply Baer's Criterion.
\end{proof}