\section{}

\begin{aexample}{Calculating Tor}{}
    Let $R=K[x,y]$. We want to find $\rm{Tor}(K,K)$.

    We have a free resolution
    \[\begin{tikzcd}
	& 1 & {(y,-x)} & 0 \\
	... 0 & R & {R^2} & R & R & K & 0 \\
	&& {(a,b)} & {(ax+by)}
	\arrow[maps to, from=1-2, to=1-3]
	\arrow[maps to, from=1-3, to=1-4]
	\arrow[from=2-1, to=2-2]
	\arrow[from=2-2, to=2-3]
	\arrow[from=2-3, to=2-4]
	\arrow[from=2-4, to=2-5]
	\arrow[two heads, from=2-5, to=2-6]
	\arrow[from=2-6, to=2-7]
	\arrow[maps to, from=3-3, to=3-4]
\end{tikzcd}\]

Tensor with $K$ and we get 
% https://q.uiver.app/#q=WzAsNSxbMSwwLCJLIl0sWzIsMCwiS14yIl0sWzMsMCwiSyJdLFs0LDAsIjAiXSxbMCwwLCIwIl0sWzAsMSwiMCJdLFsxLDIsIjAiXSxbMiwzXSxbNCwwXV0=
\[\begin{tikzcd}
	0 & K & {K^2} & K & 0
	\arrow[from=1-1, to=1-2]
	\arrow["0", from=1-2, to=1-3]
	\arrow["0", from=1-3, to=1-4]
	\arrow[from=1-4, to=1-5]
\end{tikzcd}\]
The boundary maps $\times x, \times y$ all become the zero maps. So we have the homology is exactly the module in the respective degree.
\end{aexample}
\example[]{
    As an exercise, calculate Tor of $(m,K)$ and $(K,m)$, for $m=(x,y)$. (this will be on final w.h.p)
    Verify that they are the same.
}


\subsection*{Tor and long exact sequences} 
\begin{atheorem}{Long Exact Sequence of Tor}{}
    
Let $0\to M'\to M\to M''\to 0$ exact. We have a long exact sequence 
% https://q.uiver.app/#q=WzAsMTEsWzAsMSwiXFxybXtUb3J9XjIoTScsTikiXSxbMSwxLCJcXHJte1Rvcn1eMihNLE4pIl0sWzIsMSwiXFxybXtUb3J9XjIoTScnLE4pIl0sWzAsMiwiXFxybXtUb3J9XjEoTScsTikiXSxbMSwyLCJcXHJte1Rvcn1eMShNLE4pIl0sWzIsMiwiXFxybXtUb3J9XjEoTScnLE4pIl0sWzAsMywiTSdcXG90aW1lcyBOIl0sWzEsMywiTVxcb3RpbWVzIE4iXSxbMiwzLCJNJydcXG90aW1lcyBOIl0sWzMsMywiMCJdLFsyLDAsIi4uLiJdLFswLDFdLFsxLDJdLFsyLDNdLFszLDRdLFs0LDVdLFs1LDZdLFs2LDddLFs3LDhdLFs4LDldLFsxMCwwXV0=
\[\begin{tikzcd}
	&& {...} \\
	{\rm{Tor}^2(M',N)} & {\rm{Tor}^2(M,N)} & {\rm{Tor}^2(M'',N)} \\
	{\rm{Tor}^1(M',N)} & {\rm{Tor}^1(M,N)} & {\rm{Tor}^1(M'',N)} \\
	{M'\otimes N} & {M\otimes N} & {M''\otimes N} & 0
	\arrow[from=1-3, to=2-1]
	\arrow[from=2-1, to=2-2]
	\arrow[from=2-2, to=2-3]
	\arrow[from=2-3, to=3-1]
	\arrow[from=3-1, to=3-2]
	\arrow[from=3-2, to=3-3]
	\arrow[from=3-3, to=4-1]
	\arrow[from=4-1, to=4-2]
	\arrow[from=4-2, to=4-3]
	\arrow[from=4-3, to=4-4]
\end{tikzcd}\]

\end{atheorem}

\begin{proof}
    Take a free resolution of $N$
    \[
    F_3\to F_2\to F_1\to N \to 0
    \]
    We tensor this with the exact seqeunce \[
    0\to M'\to M \to M'' \to 0
    \]
    and get the exact sequence of complexes% https://q.uiver.app/#q=WzAsMjEsWzEsMSwiRl8zXFxvdGltZXMgTSciXSxbMiwxLCJGXzNcXG90aW1lcyBNIl0sWzMsMSwiRl8zXFxvdGltZXMgTScnIl0sWzEsMiwiRl8yXFxvdGltZXMgTSciXSxbMiwyLCJGXzJcXG90aW1lcyBNIl0sWzMsMiwiRl8yXFxvdGltZXMgTScnIl0sWzEsMywiRl8xXFxvdGltZXMgTSciXSxbMiwzLCJGXzFcXG90aW1lcyBNIl0sWzMsMywiRl8xXFxvdGltZXMgTScnIl0sWzAsMSwiMCJdLFswLDIsIjAiXSxbMCwzLCIwIl0sWzQsMSwiMCJdLFs0LDIsIjAiXSxbNCwzLCIwIl0sWzEsNCwiMCJdLFsyLDQsIjAiXSxbMyw0LCIwIl0sWzEsMCwiLi4uIl0sWzIsMCwiLi4uIl0sWzMsMCwiLi4uIl0sWzksMF0sWzAsMV0sWzEsMl0sWzIsMTJdLFs1LDEzXSxbMyw0XSxbNCw1XSxbMTAsM10sWzAsM10sWzExLDZdLFsxLDRdLFs2LDddLFszLDZdLFs0LDddLFsyLDVdLFs1LDhdLFs2LDE1XSxbNywxNl0sWzgsMTddLFs3LDhdLFsxOSwxXSxbMTgsMF0sWzIwLDJdLFs4LDE0XSxbNSw0LCJsaWZ0IiwwLHsib2Zmc2V0IjotMywiY29sb3VyIjpbMzAwLDYwLDYwXSwic3R5bGUiOnsiYm9keSI6eyJuYW1lIjoiZG90dGVkIn19fSxbMzAwLDYwLDYwLDFdXSxbNCw3LCIiLDAseyJvZmZzZXQiOi0zLCJjb2xvdXIiOlszMDAsNjAsNjBdLCJzdHlsZSI6eyJib2R5Ijp7Im5hbWUiOiJkb3R0ZWQifX19XSxbNyw2LCIiLDAseyJvZmZzZXQiOi0zLCJjb2xvdXIiOlszMDAsNjAsNjBdLCJzdHlsZSI6eyJib2R5Ijp7Im5hbWUiOiJkb3R0ZWQifX19XV0=
\[\begin{tikzcd}
	& {...} & {...} & {...} \\
	0 & {F_3\otimes M'} & {F_3\otimes M} & {F_3\otimes M''} & 0 \\
	0 & {F_2\otimes M'} & {F_2\otimes M} & {F_2\otimes M''} & 0 \\
	0 & {F_1\otimes M'} & {F_1\otimes M} & {F_1\otimes M''} & 0 \\
	& 0 & 0 & 0
	\arrow[from=1-2, to=2-2]
	\arrow[from=1-3, to=2-3]
	\arrow[from=1-4, to=2-4]
	\arrow[from=2-1, to=2-2]
	\arrow[from=2-2, to=2-3]
	\arrow[from=2-2, to=3-2]
	\arrow[from=2-3, to=2-4]
	\arrow[from=2-3, to=3-3]
	\arrow[from=2-4, to=2-5]
	\arrow[from=2-4, to=3-4]
	\arrow[from=3-1, to=3-2]
	\arrow[from=3-2, to=3-3]
	\arrow[from=3-2, to=4-2]
	\arrow[from=3-3, to=3-4]
	\arrow[from=3-3, to=4-3]
	\arrow[shift left=3, color={rgb,255:red,214;green,92;blue,214}, dotted, from=3-3, to=4-3]
	\arrow["lift", shift left=3, color={rgb,255:red,214;green,92;blue,214}, dotted, from=3-4, to=3-3]
	\arrow[from=3-4, to=3-5]
	\arrow[from=3-4, to=4-4]
	\arrow[from=4-1, to=4-2]
	\arrow[from=4-2, to=4-3]
	\arrow[from=4-2, to=5-2]
	\arrow[shift left=3, color={rgb,255:red,214;green,92;blue,214}, dotted, from=4-3, to=4-2]
	\arrow[from=4-3, to=4-4]
	\arrow[from=4-3, to=5-3]
	\arrow[from=4-4, to=4-5]
	\arrow[from=4-4, to=5-4]
\end{tikzcd}\]

Where the homology of each complex is exactly the derived Tor functors. We take the long exact sequence of this short exact sequence and we are done.

The boundary map from $\rm{Tor}^{i+1}(M'',N)\to \rm{Tor}^{i}(M',N)$ is defined through the pink arrows.
\end{proof}


\begin{remark}
    I dont want to diagram chase. Just take homology functor for granted.

    Let me know if you really want the proof in the notes and I'll write it up. There is a possibility it'll be in the hw anyway.
\end{remark}

\begin{remark}
    The only part in the proof that requires special properties of tensoring is that the exactness at each level is kept (free modules are flat).
\end{remark}
%allen taught me well

\begin{quotation}
    Go to a quiet place. Put some music on, and go through the entire proof yourself. - Ananth Shankar, 2025
\end{quotation}

\corollary[]{
    Fix $M$. If $\rm{Tor}^1(M,N)=0$ for all $N$, then $M$ is flat. 
}
\begin{proof}
    Take any short exact sequence \[
    0\to N' \to N \to N'' \to 0
    \]
    the long exact sequence here preserves \[
    0\to M\otimes N'\to M\otimes N \to M \otimes N'' \to 0
    \]
\end{proof}