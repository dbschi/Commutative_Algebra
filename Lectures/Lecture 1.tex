If you see any typos, please email \href{mailto:chili2025@u.northwestern.edu}{chili2025@u.northwestern.edu}.
\section{Syllabus}
\newsection

\subsection*{Topics}
\begin{enumerate}
    \item Commutative algebra, linear algebra, tensor algebra.
    \item Rings, ideals, modules, localization, Zariski topology/spec, tesor products.
    \item Further topics include: Noether's normalization, going up and going down, completions of rings, dimension theory, Zariski's main thoerem, Nullstellensatz.
    \item Representation theory, noncommutative algebra.
\end{enumerate}
\subsection*{References}
\begin{enumerate}
    \item Atiyah and Macdonald (lots of problems here but pretty terse)
    \item Milne's notes on commutative algebra
\end{enumerate}
Both are available for free online.
\subsection*{Grades}
\begin{itemize}
    \item Midterm: $20\%$
    \item Final: $20\%$
    \item Problemsets (fortnightly): $60\%$
\end{itemize}
Ask for hints on the problemsets only for the first 9 days. Office hours Saturday 1-2 on zoom. Further OH TBD.
\section{}
\subsection*{Notation}
All rings are commutative, usually denoted $R, A, B$ and have multiplicative identity $1$.

$I,J,M,P$ denote ideals. $M$ ideals are maximal and $P$ ideals are prime.

Modules are denoted by $M, N$.

\definition{Prime Ideal}{
    An ideal $P$ is prime if \[
    xy\in P \implies x\in P \textrm{ or } y\in P.
    \]
}
\definition{Maximal Ideal}{
    An ideal $M$ is maximal if $M\subset I\implies I=R$.
}
\proposition{Maximal ideals are prime}
\begin{proof}
    We can show something stronger. We have equivalent definitions that\begin{itemize}
        \item An ideal $I$ is prime iff $R/I$ is an integral domain.
        \item An ideal $I$ is maximal iff $R/I$ is a field. 
    \end{itemize}
    A field is an integral domain so we are done.
\end{proof}
\definition{Special elements of ring}{
    Let $x\in R$. Then $x$ is \begin{enumerate}
        \item A unit, if there is $y\in R$ such that $xy=1$.
        \item A zero divisor, if there is $y\in R\backslash{0}$ such that $xy=0$.
        \item Nilpotent, if there is some $n$ such that $x^n=0$.
    \end{enumerate}
}
\begin{remark}
    The set of units form a multiplcative set. The complement of units need not form an ideal, for instance $\mathbb{Z}/6$. Similarly, the set of zero divisors need not form an ideal.
\end{remark}
\proposition{
    The set of nilpotent elements form an ideal. We call this the nilradical of $R$ and denote it $n(R)$.
}
\begin{proof}
    $0$ is nilpotent, $n(R)$ is nonempty.
    Let $x^n=0, y^m=0$. Then $(x+y)^{n+m}=0=(rx)^n$ for any $r\in R$. So the nilradical is closed under addition and multiplcation.
\end{proof}
\proposition{
    We have \[
    n(R/n(R))=\{0\}.
    \]
}
\begin{proof}
    Let $[a]$ be nilpotent in $R/n(R)$. 
    Then there exists $k$ such that $a^k\in n(R)$. But then this means $a$ is nilpotent in $R$ too. So $[a]=0$.
\end{proof}
\definition{Reduced ring}{
    $R$ is reduced if $n(R)$ is trivial.
}

Similarly we can define nilradicals based on other ideals. \definition{Nilradical}{
    Let $I\subseteq R$ be an ideal. Then the nilradical of $I$ is \[
    n(I)\defeq \{r \in R : \exists n \textrm{ s.t. } r^n\in I\}.
    \]
} 
\proposition{
    \begin{itemize}
        \item $n(I)$ is an ideal.
        \item $n(R/n(I))$ is trivial.
    \end{itemize}
}
\begin{proof}
    The same as above.
\end{proof}



